\section{Conclusions and Future Work}\label{section6}
 
This work presented M-SPLear\allowbreak ning -- an SPL that intends to support the systematic generation of m-learning applications on the Android platform. According to van der Linden et al \cite{vanderlinden07}, abstractions such as SPL allow the development process to be documented and reused systematically, contributing to the better comprehension of the target domain.

The main contribution of this work is related to the proposition of M-SPLearning and its experimental evaluation, which evidenced an improvement in the process of developing educational mobile applications through the use of SPL. In this scenario, the fault tolerance and time-to-market of such applications is essential to the end users. Therefore, there is a need of experimentally evaluating the products generated by the proposed SPL.

We experimentally evaluated the use of M-SPLear\allowbreak ning with respect to the singular software development. The obtained results were significant for the reuse approach, showing a reduction on time-to-market and a better quality in terms of number of faults when considering the software products developed with the support of variabilities. 

We also point out that the SMarty approach was crucial to the development of M-SPLear\allowbreak ning. The ease of import the SMarty Profile in a UML tool and the support provided by the SMarty Process in the application and recognition of variability have provided cost savings and better quality to the software products developed. Furthermore, with the experimental evaluations performed, improvements were applied in their elements, making SMarty a more complete and concise approach to be used. %These benefits justified its selection in our research.

%Como trabalhos futuros, pretende-se evoluir a M-SPLear\allowbreak ning com base nos insumos fornecidos a partir de sua avaliação empírica. Nesse sentido, ainda existe uma quantidade significativa de informações que podem induzir a novas linhas de pesquisa e avaliações experimentais.
As future work, we intend to evolve M-SPLear\allowbreak ning based on the inputs provided by the experimental evaluation performed. Actually, there is still a significant amount of information that can lead to new research lines and experimental evaluations. 
%Nesse sentido, novos experimentos estão sendo planejados em duas perspectivas: (i) replicação desta avaliação experimental, com o objetivo de aumentar o poder estatístico da amostra inicial; e (ii) projeto e execucão de novos experimentos, com o objetivo de avaliar a LPS e não seus respectivos produtos, como explorado neste trabalho. 
In this sense, other experiments have been planned in two perspectives: (i) replication of the experimental evaluation presented herein, in order to increase the statistical power of the initial sample; and (ii) design and execution of new experiments, aiming at evaluating SPL itself and not only its respective products, as explored in this work.

%Além disso, o modelo conceitual desenvolvido está em processo de avaliacão e modificação, considerando o domínio específico de ensino de programação. Com isso, estamos conduzindo mais estudos sobre esta adoção, possibilitando melhorias em nossa proposta e tornando-a mais adequada para adoção tanto na academia quanto na industria.

The conceptual model developed has also been evaluated and evolved, considering the specific domain of programming teaching. More studies on the adoption of M-SPLear\allowbreak ning in different contexts should enable improvements in our proposal, making it more suitable for the use in both academia and industry.

%Além disso, para que a M-SPLear\allowbreak ning possa ser avaliada em sua totalidade, todas as features elicitadas pelo catálogo de requisitos devem ser devidamente implementadas. Desta forma, um estudo empírico envolvendo a LPS propriamente dita pode ser conduzido, porque nosso experimento aferiu apenas os produtos da M-SPLear\allowbreak ning.
Moreover, for a complete evaluation of M-SPLear\allowbreak ning, all features elicited by the catalog requirements must be properly implemented. Therefore, an experimental study involving the SPL itself should be also conducted. %, since our experiment measured only M-SPLear\allowbreak ning products.

%Considerando o domínio explorado, a plataforma Android vem recebendo constantes contribuições em sua estratégia de desenvolvimento. Com isso, o aperfeiçoamento da M-SPLear\allowbreak ning deve sempre considerar a avaliação de novas ferramentas, fazendo com que a LPS evolua de acordo com as tendências do mercado.
%Considering the explored domain, the Android platform has received ongoing contributions to its development strategy. Thus, the improvement of M-SPLear\allowbreak ning must always consider the evaluation of new tools, causing the LPS evolve according to market trends.

%Para avaliação das evoluções da M-SPLear\allowbreak ning novos experimentos devem ser conduzidos, explorando vertentes relevantes para contexto educacional. Nesse sentido, avaliar os produtos gerados considerando variáveis como usabilidade e efetividade devem proporcionar resultados expressivos para este estudo. Além disso, as aplicações móveis resultantes podem ser aplicadas  em cenários reais de ensino e aprendizagem, com o objetivo de avaliar a M-SPLear\allowbreak ning aplicada ao seu domínio de usuários.
%evolutions of 

%To evaluate the M-SPLear\allowbreak ning new experiments should be conducted, exploring relevant  aspects to educational context. In this sense, evaluate the products considering variables such as usability and effectiveness should provide significant results for this study. In addition, the resulting mobile applications can be applied in real scenarios of teaching and learning, in order to evaluate the M-SPLear\allowbreak ning applied to its potential users.

%Outra perspectiva interessante está relacionada à investigação das outras estratégias de adoção propostas por. O modelo extrativo poderia ser aplicado a produtos externos a M-SPLear\allowbreak ning, com o objetivo de expandir suas features. Além disso, o modelo reativo poderia ser estudado como uma alternativa para evoluções futuras da LPS proposta.
Finally, we also intend to investigate the use of other adoption models \cite{krueger02}. For instance, the extractive model can be applied in similar products to those generated by M-SPLear\allowbreak ning, aiming at increasing the validity of similarities and variabilities specified. The reactive model can be investigated as an alternative to the evolution of the proposed SPL as well.