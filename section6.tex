\section{Conclusions and Future Work}\label{section6}
 
This work presented \texttt{M-SPLear\allowbreak ning} -- an SPL that intends to support the systematic generation of m-learning applications on the Android platform. According to van der Linden et al \cite{vanderlinden07}, abstractions such as SPL allow the development process to be documented and reused systematically, contributing to the better comprehension of the target domain.

The main contribution of this work is related to the proposition of M-SPLearning and its experimental evaluation, which evidenced an improvement in the process of developing educational mobile applications through the use of SPL. In this scenario, the fault tolerance and time-to-market of such applications is essential to the end users. Therefore, there is a need of experimentally evaluating the products generated by the proposed SPL.

We experimentally evaluated the use of \texttt{M-SPLear\allowbreak ning} with respect to the singular software development. The obtained results were significant for the reuse approach, showing a reduction on time-to-market and a better quality in terms of number of faults when considering the software products developed with the support of variabilities. 

We also point out that the SMarty approach was crucial to the development of \texttt{M-SPLear\allowbreak ning}. The ease of import the SMarty Profile in a UML tool and the support provided by the SMarty Process in the application and recognition of variability have provided cost savings and better quality to the software products developed. Furthermore, with the experimental evaluations performed, improvements were applied in their elements, making SMarty a more complete and concise approach to be used.

As future work, we intend to evolve \texttt{M-SPLear\allowbreak ning} based on the inputs provided by the experimental evaluation performed. Actually, there is still a significant amount of information that can lead to new research lines and experimental evaluations. In this sense, other experiments have been planned in two perspectives: (i) replication of the experimental evaluation presented herein, in order to increase the statistical power of the initial sample; and (ii) design and execution of new experiments, aiming at evaluating SPL itself and not only its respective products, as explored in this work.

The conceptual model developed has also been evaluated and evolved, considering the specific domain of programming teaching. More studies on the adoption of \texttt{M-SPLear\allowbreak ning} in different contexts should enable improvements in our proposal, making it more suitable for the use in both academia and industry.

Moreover, for a complete evaluation of \texttt{M-SPLear\allowbreak ning}, all features elicited by the catalog requirements must be properly implemented. Therefore, an experimental study involving the SPL itself should be also conducted

Finally, we also intend to investigate the use of other adoption models \cite{krueger02}. For instance, the extractive model can be applied in similar products to those generated by \texttt{M-SPLear\allowbreak ning}, aiming at increasing the validity of similarities and variabilities specified. The reactive model can be investigated as an alternative to the evolution of the proposed SPL as well.