\section{Introduction}

In the last years, several changes in approaches to software reuse have led to the concept of Software Product Lines (SPL), that represents perhaps the most exciting paradigm shift in software development since the advent of high-level programming languages \cite{vanderlinden07}. Thereby, companies that have developed design-by-project software for years, now focus on building and maintaining SPL and their respective variabilities. 

For an SPL to function properly, its domain must be carefully defined. If the domain is too large and the product members vary widely, the core assets will be overloaded beyond their ability to accommodate variation, production savings will be lost, and the product line will collapse. Variability management enables diversification in the portfolio of products in a given domain and requires the adoption of a well-defined systematic approach \cite{bockle05,vanderlinden07}.

From a different but related perspective, the rapid growth of information and communication technologies has favored the emergence of innovative ways for facing the shortcomings of traditional education \cite{west12}. Mobile learning (m-learning), for instance, has provided a strong interaction between learners and instructors, enabling them to actively participate in the knowledge construction process anytime and anywhere \cite{moreira2018special,kukulska05}. 

Despite the benefits provided in the context of teaching and learning, m-learning presents some problems and challenges in its adoption \cite{sharples13}. One of these problems is that the existing methods for developing software are still very generic and do not address specific aspects of mobile learning applications. Due to the diversity of platforms, technologies and pedagogical methods considered for the development of m-learning applications, a wide range of specificities can be streamlined and addressed from a reuse perspective. 

However, few studies have focused on development issues through a strategy of systematic reuse, such as SPL. \cite{bezerra09} and \cite{chen11} conducted systematic reviews and both identified gaps related to SPL adoption and variability management respectively.

The lack of studies that investigating these concepts in the domain of m-learning applications motivate us to conduct: (i) the establishment and development of an SPL such this domain \cite{falvojr14,falvojr14b}; and (ii) the evaluation of the proposed SPL. The SPL established was called \texttt{M-SPLear\allowbreak ning}, such SPL has been developed based on a concise UML-based variability management approach, named SMarty  (\textbf{S}tereotype-based \textbf{M}anagement of V\textbf{ar}iabili\textbf{ty}), which provides mechanisms to facilitate the identification and representation of variabilities \cite{marcolino13, marcolino14a, marcolino14b, bera15, marcolino2017}.

In this paper we discuss how variability improves the development of m-learning applications by taking into consideration the SMarty approach. Besides that, we evaluated the \texttt{M-SPLear\allowbreak ning} experimentally, comparing it with a singular software development methodology. 

Experimental evaluations may shed light on the identification of evidence from quality and benefits of an approach, justifying its choice. In some cases, the collected evidence may still support the correction of problems identified during the experimental evaluations and improvements in the proposals \cite{wohlin12,juristo10}.

Therefore, the experimental evaluation of \texttt{M-SPLear\allowbreak ning} is presented based on two relevant software development variables: time-to-market and number of faults. These variables can be directly influenced by the adopted development methodology and approaches that support the variabilities and commonalities management \cite{hubaux10,capilla13}. Additionally, in industry, quality and time-to-market, such as other variables, can define the success or unsuccess of the business to satisfactorily attend their clients \cite{hubaux10}.

The experiment was conducted at a real software development company, integrating academic/research proposals with practices from industry and the acquired experience of the participants. Consequently, the comparison of the methodologies, as well as the development process of the \texttt{M-SPLear\allowbreak ning}, can demystify the concept of SPL in the industry. With this, companies should be able to analyze their field of action and evaluate if a systematic reuse approach would be adequate.

The paper is organized as follows: Section \ref{section2} summarizes the background of our research, including the main characteristics of \texttt{M-SPLear\allowbreak ning}; Section \ref{section3} adresses its experimental evaluation. Section \ref{section4} discusses the lessons learned from the development and application of \texttt{M-SPLear\allowbreak ning}. Section \ref{section5} discusses the related works, and Section \ref{section6} provides conclusions and perspectives for future work.

%TODO [OK] Citar os gaps relacionados ao gerenciamento de variabilidades.
